\chapter{Vulnerability Notification Tool}
\label{chap5-vulnerability-notification-tool}
\thispagestyle{empty}

\section{Motivation}

Every day new vulnerabilities and security updates are published. Some of these vulnerabilities affect CERN websites/web servers and can be critical, therefore, it is important to learn about them as soon as possible and notify the owners of the affected resources to take necessary actions; i.e., patch their resource. The current procedure in CERN security team for responding to newly published vulnerabilities is as follows:
\begin{enumerate}
\item Getting informed about vulnerabilities from various sources by monitoring public databases, project mailing lists, security mailing lists, twitter accounts, blogs, etc.
\item Deciding if a vulnerability is worth investigating and it might affect CERN (human decision)
\item In case of vulnerabilities related to web applications, the next step is to review the output of WAD on all CERN websites and web servers, in order to get a list of resources that might be affected.
\item Sending notifications to the resource owners after making sure that the resource is infact vulnerable, i.e runs the vulnerable version or has a vulnerable configuration.
\end{enumerate}

The main goal of the vulnerability notification tool is to automate this procedure as much as possible and optimize each step of the process. 
Staying up to date with vulnerability sources and announcements is a time consuming task and we still do not know if we are monitoring the best sources. Are there other available sources that publish vulnerabilities more quickly with more accurate information and are more complete? With the current procedure we might also ignore some vulnerabilities that are important for CERN, but the public is not talking a lot about them. 
On the other hand there are so many vulnerabilities that are published every day. Figure ????\footnote{\url{http://web.nvd.nist.gov/view/vuln/statistics-results?adv_search=true&cves=on&pub_date_start_month=0&pub_date_start_year=2000&pub_date_end_month=11&pub_date_end_year=2014}} shows the trend in the number of vulnerabilities published in the last 15 years and apparently 2014 with almost 8000 vulnerabilities has been the worst year for security. Due to the ever increasing volume of public vulnerability reports CVE MITRE has even updated the CVE-ID syntax since January 2015, because the previous 4 digit format could only identify at most 9999 vulnerabilities per year.\footnote{\url {https://cve.mitre.org/cve/identifiers/syntaxchange.html}}. Even if we get informed about all vulnerabilities, it is impossible to investigate every single published vulnerability. The vulnerability notification tool tries to save the organization time by filtering out the non-critical vulnerabilities or the vulnerabilities that do not affect CERN, so that more time can be spent on remediation rather than detection.

%\section{Related Work}
%
%\subsection{Cassandra}

\section{Source Evaluation}

There are plenty of public sources that announce vulnerabilities or maintain a database of all vulnerabilities over years. The performance of the vulnerability notification tool very much depends on the quality of the source it is using. Obviously it is not possible to find a perfect source. For example there is a trade-off between the speed of publishing the vulnerabilities and accuracy of the information published; hence, it is crucial to know the needs of the organization and choose a source that fits these needs the most.

\subsection{Approach}

In order the evaluate the vulnerability sources the following evaluation factors have been considered:
\begin{enumerate}
\item \textbf{Completeness}: Does the source contain all the vulnerabilities that CERN would probably care about? 
\item \textbf{Speed}: How long since the disclosure of the vulnerability it takes until the vulnerability appears in the source?
\item \textbf{Information Quality}: What information( e.g., severity level, exploits, solutions) about a vulnerability is published on this source?
\item \textbf{Parsable Feed} the source provide parsable feed that can be downloaded/updated automatically?
\end{enumerate} 

Table \ref{table:sample_vulns} contains 6 vulnerabilities published over the last few months. These vulnerabilities have been used as samples to evaluate completeness and speed of different sources. These vulnerabilities were of great importance for CERN and can be a good indicator of how well a source fits CERN needs.
\begin{table}
\begin{center}
    \begin{tabular}{ | c | c | c| }
    
    \hline
    Vulnerability & CVE-ID & Disclosure Date 
    %& Summary 
    \\ \hline
    GitLab groups API & CVE-2014-8540 & 30.10.2015
%     & The vulnerability allows a guest user to delete the owner of a group and to assign any other member as owner through the groups API.
      \\ \hline
    Wordpress 4.0.1 Security Release & CVE-2014-9032(*)  & 20.11.2014
    % & XSS vulnerability allows remote attackers to inject arbitrary web script or HTML via unspecified vectors.[...]\footnote{\url{https://wordpress.org/news/2014/11/wordpress-4-0-1/}} 
    \\ \hline
    Drupal SQL injection
 & CVE-2014-3704
 & 15.10.2014 
 %& The vulnerability allows remote attackers to conduct SQL injection attacks via an array containing crafted keys
  \\
    \hline

 Poodle
 & CVE-2014-3566
 & 14.10.2014
 %& The SSL protocol 3.0 allows man-in-the-middle attackers to obtain clear text data via a padding-oracle attack
  \\
    \hline

Twiki Remote Code Execution
 & CVE-2014-7236
 & 09.10.2014
 %& The debugenableplugins request parameter allows arbitrary Perl code execution.  
 \\
    \hline

ShellShock
 & CVE-2014-6271
 & 24.09.2014
 %& Via certain applications, a local or remote attacker may inject shell commands, allowing local privilege escalation or remote command execution depending on the application vector. 
  \\
    \hline

    \end{tabular}
    \caption{Sample vulnerabilities}
    \label{table:sample_vulns}
   \end{center}
    \footnotesize{(*) This update fixes multiple vulnerabilities with CVE-2014-9032 CVE-2014-9033 CVE-2014-9034 CVE-2014-9035 CVE-2014-9036 CVE-2014-9037.}
\end{table}


\subsection{Vulnerability Sources}

There is a wide range of mailing lists, databases, vulnerability sources,etc  available online and each of these sources provide different kinds of information for different purposes. A part of this project was to research on these different sources and compare them in respect to CERN needs. In this section we will give a summary of the sources that were evaluated. 
 
\subsubsection{Secunia Vulnerability Intelligence Manager (VIM)} 
%a small introduction of the tool

%pros

%cons

...
I don't know if I have things to write about in this section

Although it has not been a decision factor, it is worth mentioning that Secunia VIM is reporting the vulnerabilities patch based ... blah blah 
\subsection{Conclusion}
Downloading the latest vulnerability information automatically is one of the requirements of the tool. In addition, the tool is supposed to analyze the vulnerability data to filter out the non-critical and non-related vulnerabilities, therefore, it is important that the vulnerability information is represented in a structured and parsable format. 
The only sources that provide a useful parsable feed are NVD, OSVDB and Secunia VIM. Secunia VIM focuses on vulnerability management, assuming that there is human using its portal and taking necessary actions. There is no easy way of connecting it to another tool, i.e WAD, and it is missing an API access. Although an XML description of vulnerabilities from last 72 hours can be downloaded in a semi-automatic way, i.e using tools like wget, we decided not to go for Secunia VIM, because buying the whole product and using only a non-main functionality of it was not beneficial.
Tables ??? and ??? show it clearly that OSVDB is faster, more complete and more informative than NVD. In spite of that, NVD was the source we finally used for implementation of the tool, because OSVDB is offering the API access and its feed commercially (under the name VulnDB API) and due to business reasons we were not able to acquire the license. 

\subsection{National Vulnerability Database}

	recommended by CVE MITRE
    Extends CVE MITRE by adding severity rating, fix information, etc.
    Provides a data feed in XML format. A yearly archive along with two separate files for recent and modified vulnerabilities (in an 8 day time fram). recent.xml contains only newly published vulnerabilities while modified.xml contains both newly published and modified ones. Both files are supposed to be updated every two hours.
    Although it is said that NVD gets its data from CVE MITRE, there are cases where a vulnerability is published on NVD but not CVE (look at the comparison examples)
    Curiosity: How often a vulnerability changes? In case of Drupal *****TODO***** 
    
NVD provides 2 
\subsubsection{Data Feeds}
The entire NVD can be downloaded on its web page for free and for public use. The XML vulnerability feed contains security related software flaws. Each vulnerability in the file includes a description and associated reference links from the CVE dictionary feed, as well as a CVSS base score\footnote{Common Vulnerability Scoring System is a standard measurement system for rating the severity of IT vulnerabilities}, vulnerable product configuration, and weakness categorization. The feed provides vulnerability information since 1999 (one file for each year, except that the file from 2002 includes all vulnerabilities published in 2002 and before). In addition, there is a "recent" feed, listing recently published vulnerabilities and a "modified" feed provides the list of recently published and modified vulnerabilities where "recently" is defined as the previous eight days. The feeds are updated approximately every two hours.

\section{Specifications and Implementation}
Vulnerability Notification Tool is a tool that downloads the latest "modified" feed from NVD, finds the vulnerabilities that are newly published or have been changed since its last execution and for each of these vulnerabilities reports CVE-ID, description, CVSS score, published date and time, last modified date and time, and a list of affected software (vulnerable softwares in CPE format). In addition, the tool lists the URL of the CERN websites and web servers that are likely to have this vulnerability. 
\subsection{Challenges}
\subsubsection{Downloading}
NVD keeps no history of the changes, you might miss some changes
The "modified" feed from NVD contains the latest vulnerabilities published or modified in the last 8 days. Obviously, if the tool is not run for more than eight days, it will definitely miss some updates. This issue can be fixed by looking at the yearly feed and detecting if a vulnerability information has changed in that feed since the last execution. Anyhow, it was decided that the tool will be running at least once a day and therefore there is no need for consulting the yearly feeds. 
Another point worth mentioning is that NVD keeps no record of the changes that happen in vulnerability data. It is impossible to see how a vulnerability has changed over time, because new data is always overwriting the old one. Vulnerability Notification Tool does not care about any intermediate changes in a vulnerability information between two consequent execution, but for a complete evaluation of the tool and in order to be able to simulate the tool over a period of time, we kept a local copy of the "modified" from downloaded daily since October 31\textsuperscript{th}, 2014. 
\subsubsection{Modification Date}
\subsection{and/or configuration}

\section{Product Name Matching}
\subsection{Common Platform Enumerations}
\subsubsection{Challenges}
completeness
\subsection{WAD Product Names}
\subsection{Matching Algorithms}

\subsection{Evaluation of Algorithms}
Precision and recall?
\section{Notifications}




