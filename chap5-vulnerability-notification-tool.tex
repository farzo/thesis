\chapter{Vulnerability Notification Tool}
\label{chap5-vulnerability-notification-tool}
\thispagestyle{empty}

\section{Motivation}


Every day new vulnerabilities and security updates are published. Some of these vulnerabilities affect CERN websites/web servers and can be critical, therefore, it is important to learn about them as soon as possible and notify the owners of the affected resources to take necessary actions, i.e., patch their resource. The current procedure in CERN security team for responding to newly published vulnerabilities is as follows:
\begin{enumerate}
\item Getting informed about vulnerabilities from various sources by subscribing to mailing lists, following vendor twitter accounts, etc.
\item Deciding if the vulnerability is worth investigating and it might affect CERN (human decision)
\item In case of vulnerabilities related to web applications, the next step is to review the output of WAD on all CERN websites and web servers, in order to get a list of resources that might be affected.
\item Sending notification about the vulnerability to the resource owners after making sure that the resource is in fact vulnerable, i.e runs the vulnerable version or uses vulnerable configuration.
\end{enumerate}

The main goal of the vulnerability notification tool is to automate this procedure as much as possible to ...
//descibe the problems in each step. and how the tool tries to solve them



According to [[https://www.cerias.purdue.edu/site/blog/post/2007-the-year-of-the-9999-vulnerabilities/][this post]] the number of vulnerabilities greately increase every year (when are we going to have 9,999? In 2007 it was 6,514. a fair warning to everyone using or developing CVE-compatible products). This means that it will get harder and harder to keep up with newly published vulnerabilities. A graph showing number of vulnerabilities per year can be found [[http://web.nvd.nist.gov/view/vuln/statistics-results?adv_search=true&cves=on&pub_date_start_month=0&pub_date_start_year=1997&pub_date_end_month=11&pub_date_end_year=2014][here]].



\section{Related Work}
\subsection{Cassandra}

\section{Source Evaluation}

\section{National Vulnerability Database}
\subsection{CVSS}
\subsection{Challenges}
\subsubsection{Modification Date}
\subsection{and/or configuration}

\section{Product Name Matching}
\subsection{Common Platform Enumerations}
\subsubsection{Challenges}
completeness
\subsection{WAD Product Names}
\subsection{Matching Algorithms}
\subsection{Evaluation of Algorithms}
Precision and recall?
\section{Notifications}




