\chapter{Vulnerability Notification Tool}
\label{chap5-vulnerability-notification-tool}
\thispagestyle{empty}

\section{Motivation}

Every day new vulnerabilities and security updates are published. Some of these vulnerabilities affect CERN websites/web servers and can be critical, therefore, it is important to learn about them as soon as possible and notify the owners of the affected resources to take necessary actions; i.e., patch their resource. The current procedure in CERN security team for responding to newly published vulnerabilities is as follows:
\begin{enumerate}
\item Getting informed about vulnerabilities from various sources by monitoring public databases, project mailing lists, security mailing lists, twitter accounts, blogs, etc.
\item Deciding if a vulnerability is worth investigating and it might affect CERN (human decision)
\item In case of vulnerabilities related to web applications, the next step is to review the output of WAD on all CERN websites and web servers, in order to get a list of resources that might be affected.
\item Sending notifications to the resource owners after making sure that the resource is in fact vulnerable, i.e runs the vulnerable version or has a vulnerable configuration.
\end{enumerate}

The main goal of the vulnerability notification tool is to automate this procedure as much as possible and optimize each step of the process. 
Staying up to date with vulnerability sources and announcements is a time consuming task and we still do not know if we are monitoring the best sources. Are there other available sources that publish vulnerabilities more quickly with more accurate information and are more complete? With the current procedure we might also ignore some vulnerabilities that are important for CERN, but the public is not talking a lot about them. 
On the other hand there are so many vulnerabilities that are published every day. Figure ????\footnote{\url{http://web.nvd.nist.gov/view/vuln/statistics-results?adv_search=true&cves=on&pub_date_start_month=0&pub_date_start_year=2000&pub_date_end_month=11&pub_date_end_year=2014}} shows the trend in the number of vulnerabilities published in the last 15 years and apparently 2014 with almost 8000 vulnerabilities has been the worst year for security. Due to the ever increasing volume of public vulnerability reports CVE MITRE has even updated the CVE-ID syntax since January 2015, because the previous 4 digit format could only identify at most 9999 vulnerabilities per year.\footnote{\url {https://cve.mitre.org/cve/identifiers/syntaxchange.html}} There is no doubt that it impossible to investigate every single published vulnerability, therefore, it is crucial to filter out the vulnerabilities that do not affect CERN or are not critical in an automatic way.

\section{Related Work}

\subsection{Cassandra}

\section{Source Evaluation}

There are plenty of public sources that announce vulnerabilities or maintain a database of all vulnerabilities over years. The performance of the vulnerability notification tool very much depends on the quality of the source it is using. Obviously it is not possible to find a perfect source as for example there is a trade-off between the speed of publishing the vulnerabilities and accuracy of the information published; hence, It is crucial to know the needs of the organization and choose a source that fits these needs the most.

\subsection{Approach}

In order the evaluate the vulnerability sources the following evaluation factors have been considered:
\begin{enumerate}
\item \textbf{Completeness}: Does the source contain all the vulnerabilities that CERN would probably care about? 
\item \textbf{Speed}: How long since the disclosure of the vulnerability it takes until the vulnerability appears in the source?
\item \textbf{Information Quality}: What information( e.g., severity level, exploits, solutions) about a vulnerability is published on this source?
\item \textbf{Parsable Feed} the source provide parsable feed that can be downloaded/updated automatically?
\end{enumerate} 

A couple of vulnerabilities published over the last few months have been considered to evaluate completeness and speed of different sources. These vulnerabilities were of great importance for CERN and can be a good indicator of how well a source fits CERN needs.
\begin{table}
\begin{center}
    \begin{tabular}{ | c | c | c| }
    
    \hline
    Vulnerability & CVE-ID & Disclosure Date 
    %& Summary 
    \\ \hline
    GitLab groups API & CVE-2014-8540 & 30.10.2015
%     & The vulnerability allows a guest user to delete the owner of a group and to assign any other member as owner through the groups API.
      \\ \hline
    Wordpress 4.0.1 Security Release & CVE-2014-9032(*)  & 20.11.2014
    % & XSS vulnerability allows remote attackers to inject arbitrary web script or HTML via unspecified vectors.[...]\footnote{\url{https://wordpress.org/news/2014/11/wordpress-4-0-1/}} 
    \\ \hline
    Drupal SQL injection
 & CVE-2014-3704
 & 15.10.2014 
 %& The vulnerability allows remote attackers to conduct SQL injection attacks via an array containing crafted keys
  \\
    \hline

 Poodle
 & CVE-2014-3566
 & 14.10.2014
 %& The SSL protocol 3.0 allows man-in-the-middle attackers to obtain clear text data via a padding-oracle attack
  \\
    \hline

Twiki Remote Code Execution
 & CVE-2014-7236
 & 09.10.2014
 %& The debugenableplugins request parameter allows arbitrary Perl code execution.  
 \\
    \hline

ShellShock
 & CVE-2014-6271
 & 24.09.2014
 %& Via certain applications, a local or remote attacker may inject shell commands, allowing local privilege escalation or remote command execution depending on the application vector. 
  \\
    \hline

    \end{tabular}
    \caption{Sample vulnerabilities}
   \end{center}
    \footnotesize{(*) This update fixes multiple vulnerabilities with CVE-2014-9032 CVE-2014-9033 CVE-2014-9034 CVE-2014-9035 CVE-2014-9036 CVE-2014-9037.}
\end{table}


now talk about the sources you considered and at the end the comparisons in the table format. 
\begin{enumerate}
\item \textbf{Drupal SQL injection vulnerability}
\item \textbf{Poodle}
\item \textbf{Twiki remote code execution}
\item \textbf{ShellShock}
\end{enumerate}

\section{National Vulnerability Database}
\subsection{CVSS}
\subsection{Challenges}
\subsubsection{Modification Date}
\subsection{and/or configuration}

\section{Product Name Matching}
\subsection{Common Platform Enumerations}
\subsubsection{Challenges}
completeness
\subsection{WAD Product Names}
\subsection{Matching Algorithms}
\subsection{Evaluation of Algorithms}
Precision and recall?
\section{Notifications}




