\chapter{Conclusion and Future Work}
\label{conclusion-and-future-work}
\thispagestyle{empty}

This project introduced two new tools for detection of vulnerable web applications. The approaches used in these tools are different from common web scanners, such as skipfish. In this project we focused more on detecting vulnerable applications as soon as possible by monitoring publicly available vulnerability information. Once we are aware of the vulnerability, we can take two approaches to detect vulnerable resources. If the vulnerability is specific to a software or product, VNT described in chapter \ref{vulnerability-notification-tool} detect resources that use those products and report them as probable vulnerable resources. In cases that the vulnerability is not specific to a product, or detecting that a resource is using that product is impossible



possible improvements:

in mapping algorithm use string distances

looking at versions when finding resources

vnt extensions like forgetting,etc. 

scanner??


+++++
If there is new resource detected at CERN: we can check in our archive and see if the new resource is affected by old vulnerabilities -> less likely. but atleast check for the recent vulnerbailities. A part of the firewall opening? 
another scope. not responding to new vulnerabilities but making sure a new resource is not vulnerable
\\
when finding matches (resources) care about the version as well



