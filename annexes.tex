\chapter{Annexes}
\label{app}


\section{VNT Evaluation Methode}
\label{vnt_evaluation_method}
Once we have the output of the tool in JSON format, it is easy to convert it to other formats to do an analysis of the results or obtain some statistics. 
\subsection{Calculating the Number of Vulnerabilities}
The python script \texttt{vnt\_format.py} receives the JSON description of VNT results as an input and generates an output in text format describing each vulnerability in one line. This text output can be used by various command line tools, such as grep, cut, etc., to obtain statistics. 
Each line is in the following format:
\begin{framed}

\texttt{c|n;%published\_date;last\_modified\_date;reported\_date;cve\_id;nvd\_description\_url;
cvss\_score;%cpe\_names;
wad\_names
%;
%changed\_field\%old\_value\%new\_value
}

\end{framed}
The first component describes if the vulnerability is new (n) or has changed (c). If there are multiple entries in one component, i.e multiple WAD names, entries are separated by comma.
\\
For Example:
\begin{framed}
\texttt{c;6.8;Red Hat,Fedora,JBoss Web,OpenSSL}
\end{framed}
describes a vulnerability that has changed, has a CVSS score of 6.8 and matches with WAD names `Red Hat', `Fedora', `JBoss Web' and `OpenSSL'.
The following command will output the number of new vulnerabilities and changes in vulnerabilities using the TEXT description of all the results. 
\begin{framed}
\begin{verbatim}
$ cat *.txt| cut `-d;' -f1| sort| uniq -c
\end{verbatim}
\end{framed}
In order to get the critical vulnerabilities one can use the following command:
\begin{framed}
\begin{verbatim}
$ cat *.txt| grep `;\([6-9]\|10\)\.'| cut `-d;' -f1| sort|\
  uniq -c
\end{verbatim}
\end{framed}
It is also important to know the number of vulnerabilities that affect CERN. For this purpose WAD can be used to generate a file with all WAD names used at CERN (one name per line) and then using the following command can give us the same statistics as before, but only considering the vulnerabilities that affect CERN.
\begin{framed}
\begin{verbatim}
$ cat *.txt| grep -f wad_names_at_cern.txt| cut `-d;' -f1|\
  sort| uniq -c
\end{verbatim}
\end{framed}
\begin{framed}
\begin{verbatim}
$ cat *.txt| grep -f wad_names_at_cern.txt|\
  grep `;\([6-9]\|10\)\.'| cut `-d;' -f1| sort|\
  uniq -c
\end{verbatim}
\end{framed}
%
%
Do you want to talk about how you found number of emails per product?


